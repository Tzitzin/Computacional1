\documentclass{article}
\usepackage[spanish]{babel}
\usepackage[utf8x]{inputenc}
\usepackage{setspace}
\usepackage{graphicx}
\usepackage{enumerate}

\begin{document}

\begin{doublespace}

Alumna: Chávez Gutiérrez Yanneth Tzitzin\\
Instrucciones: escriba un documento en LaTeX, donde responda libremente las siguientes preguntas de reflexion:

\begin{enumerate}
\item ¿Cual es tu primera impresión de uso de LaTeX?\\
Me parece que LaTeX en una manera muy formal de presentar algun trabajo. Cuesta un poco al principio entender que comandos utilizarás para aplicar lo que facilmente aplicabas en Word, pero vale la pena cuando lo ves estructurado y puesto ya en PDF. 

\item ¿Qué aspectos te gustaron más?\\
Yo creo que me gustó el hecho de saber qué es lo que esta detras de un click por ejemplo en Word o en cualquier paquete de microsoft, entender maso menos los comandos que se necesitan para efectuar esas acciones. Y el estar entendiendo cada comando que utilizas, ya cuando quedó en tu mente el proceso que tienes que seguir para obtener lo que quieres en tu presentacion de proyecto. 

\item ¿Qué no pudiste hacer en LaTeX?\\
No sé aún que más provecho le falte sacar, yo se que LaTeX cuenta con muchas más funciones, pero dentro de lo que más batalle fue tener que estar nombrando uso de paquetes para poder llevar acabo funciones especiales podria decirse. 

\item En tu experiencia, comparado con otros editores, ¿cómo se compara LaTeX?
No es más facil de usar que otros editores, pero una vez dominado yo creo que tiene muchas más ventajas, pues en mi punto de vista es más presentable y formal, comparado con los resultados de otros editores. 

\item ¿Qué es lo que mas te llamó la atención en el desarrollo de esta actividad?\\
Ya habia trabajado con algun otro editor en linea de LaTeX, pero usaba plantillas y ya tenian incluido los paquetes etcetera. Para esta actividad decidí empezar de cero para tener en mente todo lo que se utiliza en un principio y fue lo que más me llamo la atencion en general. Qué puede ser dificil empezar un documento en blanco, pero ya con los paquetes nombrados todo se hace mucho más facil. 

\item ¿Qué cambiarías en esta actividad?\\
En general nada, se nos proporciono buena informacion base acerca de LaTeX y sobre los comandos a utilizar. Con eso fue suficiente, y me gusto mucho el tema a investigar también. 

\item ¿Que consideras que falta en esta actividad?\\
Como dije anteriormente, me fue suficiente con la informacion dada. Aunque tal vez me hubiera gustado ver en clase a grandes rasgos lo que es LaTeX y que se puede llevar a hacer con el. Una pequeña introducción desde lo más simple que se puede construir con LaTeX, hasta las cosas ya más complejas. 

\item ¿Puedes compartir alguna referencia nueva que consideras util y no se haya contemplado? \\
La única referencia extra que utilice fue la de los paquetes de LaTeX, no use solo un link, pero me hubiera gustado encontrar en linea alguna liga con la mayoria de los paquetes, y una pequeña descripcion de lo que se hace con ellos. 

\item ¿Algún comentario adicional que desees compartir?\\
Simplemente que espero podamos hacer más actividades usando LaTeX, pues me gustaria aprender bien a utilizarlo. De la manera en la que lo hice y si se puede aun con cosas más complejas que se puedan hacer usando este editor. 

\end{doublespace}

\end{document}
