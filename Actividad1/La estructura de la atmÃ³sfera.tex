\documentclass{article}
\usepackage[spanish]{babel}
\usepackage[utf8x]{inputenc}
\usepackage{setspace}

\usepackage{graphicx}
\usepackage{enumerate}
\begin{document}
\begin{center}
\textsc{\LARGE La estructura de la atmósfera}
\linebreak
%%%%%%%%%%%%%%%%%%%%%%%%%%%%%%%%%%%%%%%%%%%%%%%%%%%%%%%%%%%%%%%%%%%%%%%%%%%%%%%%%%%%%%%%
\begin{figure}[!ht]
\begin{center}
\includegraphics[width=1.0\textwidth]{atm.jpg}
\end{center}
\end{figure}
%%%%%%%%%%%%%%%%%%%%%%%%%%%%%%%%%%%%%%%%%%%%%%%%%%%%%%%%%%%%%%%%%%%%%%%%%%%%%%%%%%%%%%%%%%%%%%%%

\textsc{Chávez Gutiérrez Yanneth Tzitzin}
\linebreak

\textsc{Física computacionál}
\linebreak

\textsc{Profesor: Carlos Lizárraga Celaya}
\linebreak

\textsc{30 enero 2017}
\end{center}


\pagebreak
\section{¿Qué se abordará en este ensayo?}
\begin{doublespace}
Este ensayo será breve. Hablaremos de la composición de la atmósfera. Sus capas y poco de lo que se lleva a cabo en ellas. Después de ello, nos centraremos en el tema de cómo es que se forman las auroras boreales en los polos de nuestro planeta. Y cómo interviene la atmósfera en este proceso.
\end{doublespace}
\section{Inroducción}
\begin{doublespace}
Existen muchas preguntas en la actualidad, que pudieran parecer simples. Pero a veces nos es dificil explicarlas en pocas palabras o tal vez dificil encontrar una respuesta inmediata. Una de ellas, y de las primeras preguntas que tal vez nos hacemos al ser pequeños es: ¿Por qué el cielo es de color azul? Pareciera una pregunta inoscente, pero detras de ella hay toda una investigacion cientifica que llegó a la conclusion del porque nuestro cielo es azul. 
Principalmente se debe a la composicion de nuestra atmosfera terreste, y cómo llega a interactual con los rayos de sol. \\
La luz del sol en su escencia es blanca, y viaja 149.6 millones km hasta nuestro planeta, y no es hasta cruzar la atmosfera donde ocurre todo esto. La luz solar, choca con las particulas que la componen y se disperza a todas direcciones. El nitrogeno y el oxigeno son algunos de los elementos que conforman la atmosfera, estos principalmente se encargan de dispersar más el color azul y violeta. Dejando pasar los tonos naranjas y rojos en lineas rectas. Es por eso que nuestro cielo se ve azul. Por la rica concentración de nitrógeno y oxigeno en la atmósfera.  \\
Pero no solo de eso esta compuesta la atmósfera, hay varias partes que la componen y suceden cosas interesantes dentro de ella, que nos permiten llegar a hacer predicciones meteorológicas, hasta lo más escencial: Darle oportunidad a la vida en el planeta tierra. 
\end{doublespace}
\pagebreak

\section{La estructura de la atmósfera}
\begin{doublespace}
La tierra esta rodeada por una capa de aire llamada atmósfera, sin ella no podría existir la vida en la tierra, o al menos como la conocemos hoy en día. La atmósfera juega un papel muy importante para nosotros los humanos, pues a parte de darnos las condiciones para la vida, nos ha ayudado a desarrollar tecnologías y poder sacar provecho a fenómenos que ocurren en ella. En este ensayo, se tratará de explicar un poco acerca de su estructura, y el papel que juega la atmósfera en la proyección de lo que conocemos como auroras boreales. \\
Para ir en orden, hablaremos un poco de la estructura atmosférica, de qué son las auroras, cómo es que se forman y cómo interviene la atmósfera en este proceso. 


\subsection{Estructura y composición}
La atmósfera se compone de cuatro capas: la tropósfera que es donde nos desarrollamos como humanos, la estratósfera donde se encuentra la capa de ozono la cual es indispensable,pues ayuda principalmente a absorber las radiaciones ultravioletas nocivas para la vida en la tierra, la mesósfera en la cual la temperatura desciende marcadamente al ir aumentando la altura, y la termósfera la capa más alta, donde hay aire caliente pero muy ligero. La tropósfera es la capa más baja de la atrmósfera, es donde nosotros vivimos, y donde se genera lo que conocemos como clima (los fenómenos meteorológicos). En la siguiente página se mostrará una imagen que representa las capas ya mencionadas. Podemos ver el orden en que se encuentran, y la distancia que existe entre ellas. 
%%%%%%%%%%%%%%%%%%%%%%%%%%%%%%%%%%%%%%%%%%%%%%%%%%%%%%%%%%%%%%%%%%%%%%%%%%%%%%%%%%%%%%%%
\begin{figure}[!ht]
\begin{center}
\includegraphics[width=0.5\textwidth]{173340_jpg_1.JPG}
\caption{Imagen de la estructura de la atmósfera. Referencia: quimicaenelambiente.blogspot.mx}
\end{center}
\end{figure}
%%%%%%%%%%%%%%%%%%%%%%%%%%%%%%%%%%%%%%%%%%%%%%%%%%%%%%%%%%%%%%%%%%%%%%%%%%%%%%%%%%%%%%%%%%%%%%%%

\end{doublespace}
\pagebreak
\section{¿Qué es una aurora boreal, y cómo interviene la atmósfera?}
\begin{doublespace}
Las auroras son fenómenos luminiscentes que se producen ocacionalmente en el cielo nocturno de las zonas polares (aunque también son observables fuera de estas zonas a veces.) Estos espectáculos lumínicos se dan por las noches en los polos, manifestando una danza de colores sobre el cielo nocturno. \\
%%%%%%%%%%%%%%%%%%%%%%%%%%%%%%%%%%%%%%%%%%%%%%%%%%%%%%%%%%%%%%%%%%%%%%%%%%%%%%%%%%%%%%%%
\begin{figure}[!ht]
\begin{center}
\includegraphics[width=0.5\textwidth]{aurora-boreal.jpg}
\caption{Imagen de auroras boreales}
\end{center}
\end{figure}
%%%%%%%%%%%%%%%%%%%%%%%%%%%%%%%%%%%%%%%%%%%%%%%%%%%%%%%%%%%%%%%%%%%%%%%%%%%%%%%%%%%%%%%%%%%%%%%%
\pagebreak

Pero ¿cómo es que se forman las auroras? La aurora es causada por partículas cargadas, como electrones y protones, que golpean nuestra atmosfera. Estas particulas cargadas fluyen en nuestro campo magnético y despues de ese proceso, ellas mismas dan energia a el oxigeno y el nitrogeno en la atmósfera. Esta energia es despues liberada en forma de luz. El verde es el color principal para las auroras boreales producidas que se generan aproximadamente a 100 kilometros de altitud. Y cuando las partículas que logran entrar a la atmósfera son muy energéticas, se genera especialmente luz del color rojo. Podemos darnos cuenta que gracias a la atmósfera y sus componentes, nuevamente mencionado como en la introducción. Eso es lo que le da los colores tan bonitos que podemos ver ya sea a algún atardecer para cualquier parte del mundo, o hasta las auroras boreales producidas en los polos de nuestro planeta. \\ 
La atmósfera no solo produce este tipo de cosas, se llevan a cabo muchos fenómenos más de los que no se hablaran en este ensayo. Pero hay otras cosas igual o más interesantes a saber de ella, pues no muchas veces viene a nuestra cabeza pensar en la atmósfera, pero es esencial para la formación de la vida como la conocemos. 

\end{doublespace}
%%%%%%%%%%%%%%%%%%%%%%%%%%%%%%%%%%%%%%%%%%%%%%%%%%%%%%%%%%%%%%%%
\pagebreak
\section{Referencias}
\begin{doublespace}
\begin{enumerate}

\item Geophysical Institute, University of Alaska Fairbanks. (desconocido). The aurora index. enero 2017, de Geophysical Institute, University of Alaska Fairbanks Sitio web: http://www2.gi.alaska.edu/ScienceForum/aurora.html


\item Muy interesante. (2014). ¿Por qué el cielo es azul?. enero 2017, de Muy interesante Sitio web: http://www.muyinteresante.com.mx/preguntas-y-respuestas/14/06/11/cielo-azul/

\item Sebastian Rossi. (2012). ¿Cómo se forma una aurora boreal?. enero 2017, de VIX Sitio web: http://www.vix.com/es/btg/curiosidades/2011/05/04/como-se-forma-una-aurora-boreal

\item Arturo Covarrubias. (2014). Características físico-químicas y fenómenos de la estructura atmosférica.. enero 2017, de blog de Arturo Sitio web: http://arturo-covarrubias.blogspot.mx/2014/02/caracteristicas-fisico-quimicas-y.html

\item Department of physics, University of Oslo. (unknown). How does auroras form?. enero 2017, de UiO Sitio web: https://www.youtube.com/watch?v=L_k92H7KQAg



\end{enumerate}
\end{doublespace}

\end{document}
